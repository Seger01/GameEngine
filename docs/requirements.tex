% This is a simple sample document.  For more complicated documents take a look in the exercise tab. Note that everything that comes after a % symbol is treated as comment and ignored when the code is compiled.

\documentclass{article} % \documentclass{} is the first command in any LaTeX code.  It is used to define what kind of document you are creating such as an article or a book, and begins the document preamble

\usepackage{amsmath} % \usepackage is a command that allows you to add functionality to your LaTeX code
\usepackage{booktabs} % For better looking tables
\usepackage{geometry} % To adjust margins
\usepackage{tabularx} % For tables that auto-adjust to the page width
\usepackage{longtable} % For tables across multiple pages

\title{Requirements} % Sets article title
\author{Sean Groenenboom \and Seger Sars \and Siem Vermeulen \and Angel Villanueva \and Ronan Vlak} % Sets authors name
\date{\today} % Sets date for date compiled
\newpage

% The preamble ends with the command \begin{document}
\begin{document} % All begin commands must be paired with an end command somewhere
\maketitle % creates title using information in preamble (title, author, date)
\newpage

\section{Requirements}
\centering
\begin{longtable}{|p{0.14\textwidth}|p{0.15\textwidth}|p{0.11\textwidth}|p{0.16\textwidth}|p{0.44\textwidth}|}
    \hline
    \textbf{Requirement Number} & \textbf{Requirement Type (MoSCoW)} & \textbf{Userstory} & \textbf{Requirement Name}         & \textbf{Definition of Done}                                                                                                                                 \\ \hline
    \endfirsthead

    \hline
    \textbf{Requirement Number} & \textbf{Requirement Type (MoSCoW)} & \textbf{Userstory} & \textbf{Requirement Name}         & \textbf{Definition of Done}                                                                                                                                 \\ \hline
    \endhead

    \hline \multicolumn{5}{r}{\textit{Continued on the next page}}                                                                                                                                                                                                                          \\ \hline
    \endfoot

    \hline
    \endlastfoot

    M1                          & Must Have                          & U01                & Levels support                    & The engine shall support multiple regions (levels) in which the game can be played.                                                                         \\ \hline
    M2                          & Must Have                          & U02                & Scenes support                    & The engine shall support multiple scenes (sections in the game which are significantly different from each other e.g. menu, gameplay, credits).             \\ \hline
    M3                          & Must Have                          & U03                & Keyboard support                  & The engine can use inputs from all keys of a 75\% size keyboard to use in the game. \textit{\textbf{Is 75\% reasonable? Or should we pick something else?}} \\ \hline
    M4                          & Must Have                          & U03                & Mouse support                     & The engine can use the following inputs from a computer mouse: horizontal and vertical movement, left, right and middle mouse button and scroll wheel.      \\ \hline
    M5                          & Must Have                          & U04                & FPS counter                       & The engine shall provide a FPS counter in the top right corner of the screen, which can be toggled on and off                                               \\ \hline
    M6                          & Must Have                          & U05                & Object creation                   & A game object can be created using a single method call                                                                                                     \\ \hline
    M7                          & Must Have                          & U05                & Object deletion                   & A game objet can be removed using a single method call                                                                                                      \\ \hline
    M8                          & Must Have                          & U06                & Game speed increase               & The game engine can be sped up from its default speed up to 10 times, in steps of 0.1.                                                                      \\ \hline
    M9                          & Must Have                          & U06                & Game speed increase button        & The game engine can be sped up using the pg up button on the keyboard.                                                                                      \\ \hline
    M10                         & Must Have                          & U06                & Game speed decrease               & The game engine can be slowed down from its default speed up to 10 times, in steps of 0.01.                                                                 \\ \hline
    M11                         & Must Have                          & U06                & Game speed decrease button        & The game engine can be sped up using the pg down button on the keyboard.                                                                                    \\ \hline
    S1                          & Should Have                        & U106               & Online or LAN multiplayer support &                                                                                                                                                             \\ \hline
    S2                          & Should Have                        &                    & FPS can be fixed at a maximum     &                                                                                                                                                             \\ \hline
\end{longtable}
\newpage

\end{document} % This is the end of the document
