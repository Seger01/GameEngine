\section{Networking}
\label{sec:networking}
The engine will support multiplayer.
As the customer wants the engine to follow the Unity engine API, the multiplayer system wil follow the multiplayer package of unity \textbf{Netcode}.

\subsection{Network library}
Multiple libraries have been considered, for example ENet, RakNet or Boost.
The uses library is SLikeNet. This library is a continuation of RakNet, which is a well known and well documented library.
The library is also easy to use and has a lot of features that are useful for the engine.

\subsection{Class diagram}

\begin{figure}[H]
    \centering
    \includegraphics[width=\textwidth]{networkingClassDiagram.png}
    \caption{class diagram of the networking system}
    \label{fig:networkingClassDiagram}
\end{figure}
In \autoref{fig:networkingClassDiagram} the proxy classes are classes that are defined in a different class diagram.

A central manager named NetworkManager is used to manage the networking features.
An application can either be a server, a client or both and is then named a host.
To send information over the network a NetworkObject is used.
This object inherits Component class and can be added to a GameObject to define that this object can be sent over the network.
NetworkBehaviour inherits the behaviour class and can be used to send specific information over the network.
The NetworkTransform is used as a standard implemenation of sending the transform information over the network.
If a game programmer wants more control of what information is send when it should send this information with the NetworkBehaviour.

To send information over the network a NetworkVariable must be used.
This is a template class that defines information and how that information can be serialized.

ID's for networking:
Message ID :defines what kind of message is send
NetworkObject ID: defines the unique identifier of a networkObject. Only server may make this ID
Prefab ID: Things that can be spawned over the network must be defined as a prefab and the manager gives this a unique ID
GamePacketID: when a game dev makes a concrete implemenation of a GamePacket it must add this to a factory which gives it a unique ID that can be used by both server and client 

Spawning new object across network:
Client must send a request to the server to spawn a new object. The server will then spawn the object and send the information to all clients. All clients will then spawn the object.
message info from client: Message ID (spawn object), Prefab ID (which object)
message info from server: Message ID (spawn object), Prefab ID (which object), NetworkObject ID (unique ID of the object)

Sending object information:
Cient and server can send info about objects
message info client: Message ID (object info), NetworkObject ID (unique ID of the object), GamePacketID (unique ID of the packet), data (the data that is send)
message info server: Message ID (object info), NetworkObject ID (unique ID of the object), GamePacketID (unique ID of the packet), data (the data that is send)

NetworkVariable:
Template class which expects a GamePacket derived class. This class automatically syncs data between clients and server. The game programmer must set the data within the GamePacket. The networkmanager will check if data is changed and send this data over the network.
