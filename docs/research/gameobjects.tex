\section{GameObjects(Ronan)}

\subsection{GameObjects in a game engine}
GameObjects are the fundamental entities that represent characters, props, lights, enemies and scenery.
GameObjects are essentially containers for components, which implement the GameObject's functionality.
\\\\
\noindent A few examples of components a GameObject might have are:
\begin{itemize}
    \item \textbf{Transform component:}
          A component handling the position, rotation and scale of a GameObject.
    \item \textbf{Render component:}
          A component to display a GameObject.
    \item \textbf{Physics component:}
          A component to add physics to a GameObject, for example velocity and mass.
    \item \textbf{Script component:}
          A component to define the behaviour of a GameObject.
    \item \textbf{Collider component:}
          A component to detect and respond to collisions with the GameObject.
\end{itemize}

\noindent GameObjects are always part of a scene and can interact with each other through the added components.

\subsection{Adding components to game objects}
A GameObject is just an empty vessel that needs components to be attached to it.
To add components to a GameObject, the following steps have to be taken:
\begin{enumerate}
      \item \textbf{Defining components:}
            Components are derived from a base or interface class to ensure each component is managed in a uniform way.
            Each component will hve a specialized role, like rendering or physics.
      \item \textbf{Attaching components:}
            When a component is added to a GameObject, the engine ensures that the component integrates correctly with 
            the GameObject and other components that are already attached. Some components need another component to function.
            An example of this is the Collider component, which needs a Transform component.
      \item \textbf{Limiting component types:}
            GameObjects can have multiple components, but most components should be limited to one per game object.
            An example of this is the Physics component, which should be limited to one component to prevent conflicting behaviour.
\end{enumerate}

\noindent Adding components to GameObjects this way also includes the ability to add or remove components dynamically.
This opens up the possibility for changes in a GameObject's behaviour during gameplay.
\\
Removing components is also important for GameObjects.
The engine should be able to detach and dispose of components when they are no longer needed.
This will lead to less resource usage.