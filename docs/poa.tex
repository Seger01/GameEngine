% This is a simple sample document.  For more complicated documents take a look in the exercise tab. Note that everything that comes after a % symbol is treated as comment and ignored when the code is compiled.

\documentclass{article} % \documentclass{} is the first command in any LaTeX code.  It is used to define what kind of document you are creating such as an article or a book, and begins the document preamble

\usepackage{amsmath} % \usepackage is a command that allows you to add functionality to your LaTeX code
\usepackage{booktabs} % For better looking tables
\usepackage{geometry} % To adjust margins
\usepackage{tabularx} % For tables that auto-adjust to the page width

\title{Plan of Attack} % Sets article title
\author{Sean Groenenboom \and Seger Sars \and Siem Vermeulen \and Angel Villanueva \and Ronan Vlak} % Sets authors name
\date{\today} % Sets date for date compiled
\newpage

% The preamble ends with the command \begin{document}
\begin{document} % All begin commands must be paired with an end command somewhere
\maketitle % creates title using information in preamble (title, author, date)
\newpage

\tableofcontents % creates a table of contents
\newpage

\section{work division}
sean: Quality
siem: General planning
Seger:Project
Angel: project orginisation, risk analysis
Ronan:Document inventory, communication
\newpage

\section{version control}
\newpage

\section{Project}
The following optional user stories shall be implemented
\begin{itemize}
    \item Multiplayer / internet connection
\end{itemize}
\subsection{problem analysis}
\subsection{scope}
\subsection{goal}
\subsection{result}
\newpage

\section{General planning} % creates a section
Per week vastleggen wat er gedaan moet worden (zie papierwerk docu op BS)?

\begin{tabularx}{\textwidth}{|X|X|X|X|}
    \hline
    \textbf{Week number} & \textbf{Activity}                              & \textbf{Finished this week}                    \\ \hline
    1                    & Begin working on plan of attack                & First plan of attack draft                     \\ \hline
                         & Begin researching various potential features   & Research document draft (1 feature per member) \\ \hline
    2                    & Conduct further research on potential features &                                                \\ \hline
                         & Determine future tasks based off research      &                                                \\ \hline
                         & Begin work on first POC                        &                                                \\ \hline
    3                    & Should Have                                    &                                                \\ \hline
    4                    & Should Have                                    &                                                \\ \hline
    5                    & Should Have                                    &                                                \\ \hline
    6                    & Should Have                                    &                                                \\ \hline
    7                    & Should Have                                    &                                                \\ \hline
    8                    & Should Have                                    &                                                \\ \hline
    9                    & Should Have                                    &                                                \\ \hline
    10                   & Should Have                                    &                                                \\ \hline
    11                   & Should Have                                    &                                                \\ \hline
    12                   & Should Have                                    &                                                \\ \hline
    12                   & Should Have                                    &                                                \\ \hline
    14                   & Should Have                                    &                                                \\ \hline
    15                   & Should Have                                    &                                                \\ \hline
    16                   & Should Have                                    &                                                \\ \hline
    17                   & Should Have                                    &                                                \\ \hline
    18                   & Should Have                                    &                                                \\ \hline
    19                   & Should Have                                    &                                                \\ \hline
    20                   & Should Have                                    &                                                \\ \hline
\end{tabularx}
\label{tab:requirements}
\newpage

\section{Document inventory}
Welke docus zijn er en wat staat daarin?
\newpage

\section{Communication}
Hoe wordt er onderling gecommuniceerd (Whatsapp)
Hoe vaak komt de groep samen? Op vaste momenten?
\newpage

\section{Quality}
To ensure the quality of the project the following points are applied.
\subsection{Documentation}
The following points are applied to ensure the documentation’s quality:
\begin{itemize}
    \item All documents are written in English.
    \item All documents are written in the present perfect
    \item All diagrams are in the UML style.
    \item Headers in the research document do not contain questions
    \item No choices are made in the research document
    \item latex is used as default text editor. This is used so git can be used as version conrol.
    \item plantuml is used for UML diagrams. This is used so git can be used as version conrol.
\end{itemize}
\subsection{Technical}
the following points are applied to ensure the technical quality of the project:
\begin{itemize}
    \item Code is written in english.
    \item Code is written in c++23.
    \item Doxygen is used for code documentation, except for the setters and getters
    \item Codestandard is written in ./codestandard.tex.
    \item Text editor of choice can be used.
    \item Cmake is used to build the code.
    \item Clangformat is used for code formatting.
    \item cpp check is used for code checking.
    \item valgrind is used for memory leak checking.
    \item gtest testing framework is used for testing.
    \item Unit test are written for every function except for setters without validation and getters.
    \item Third party libraries must be discussed with the whole team.
    \item One person must check git commits before merging.
    \item Only code which can be compiled may be commited to git.
\end{itemize}

\subsection{tools}
the following tools are used to ensure the quality of the project:
\begin{itemize}
    \item Git is used for version management.
    \item Github is used to store the git repo.
\end{itemize}
\newpage

\section{project orginisation}
\newpage

\section{Risk analysis}
\newpage

\end{document} % This is the end of the document


